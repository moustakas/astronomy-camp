WISE (Wide-Field Infrared Survey Explorer) is a $40$-cm space
telescope that has mapped more than $95\%$ of the sky with
unprecedented sensitivity (about 500 times more sensitive than IRAS)
at $3.4$, $4.6$, $12$, and $22$~\micron.  In its brief, but brilliant
one-year mission, WISE has observed more than $200$ million Galactic
and extragalactic sources in this important wavelength regime.  The
survey has also discovered thousands of new solar system objects
(asteroids, comets, etc.), brown dwarfs, and distant, dusty,
star-forming galaxies.

By surveying the whole sky, WISE has unveiled thousands of rare and
unusual objects by virtue of their atypical mid-infrared colors.
However, spectroscopic follow-up is required to ascertain the nature
and physical properties of these objects.  One especially poorly
understood class of objects that WISE has observed is metal-poor
blue-compact star-forming galaxies with unusual mid-infrared colors.
The archetype galaxy in this class is SBS0335-052, whose $[3.6]-[4.5]$
color based on observations with the {\em Spitzer Space Telescope} is
more than an order-of-magnitude redder than comparably luminous,
metal-poor galaxies (Houck et al. 2004, ApJS, 154, 211; Engelbracht et
al. 2008, ApJ, 678, 804).  Before WISE, there were only two other
examples of this class of object, Haro~11 and SHOC~391.  Therefore,
the physical origin of the unusual mid-infrared spectra of these
objects remains a mystery.

In collaboration with the WISE team, we propose to use the B\&C
spectrograph at the Bok $2.3$-m telescope to obtain
intermediate-resolution ($\sim8$~\AA{} FWHM) optical ($3600-6900$~\AA)
spectroscopy of a sample of candidate ``extreme'' low-metallicity
star-forming galaxies selected on the basis of their unusual
mid-infrared colors.  The principal objective of these observations
will be to spectroscopically confirm these objects as low-metallicity
galaxies for future follow-up studies.  However, with the anticipated
signal-to-noise ratio ($>20$ in the H$\beta$ emission line) we will
also be able to estimate their gas-phase metallicitiies, ionization
parameters, optical dust extinction, and star-formation rates.  
